\documentclass[a4paper, 10pt]{article}

\usepackage[utf8]{inputenc}
\usepackage[T1]{fontenc}
\usepackage[french]{babel}
\usepackage[top=3cm, bottom=3cm, left=2cm, right=2cm]{geometry}
\usepackage{graphicx}
\usepackage{amsmath}
\usepackage{amssymb}
\usepackage{fourier}

\title{Système d'exploitation \\ TP supplémentaire \\ Protocole experimental}
\author{Enzo CADONI}
\date{}

\begin{document}
  \maketitle
  \newpage

  \section{Préambule}
    Ce protocole s'appuiera sur l'utilisation de l'exécutable \verb+ benchmark +
    situé dans le repertoire \verb+ bin +. \\

    Cet exécutable permet de tester tous les algorithmes avec des paramètres
    fixés passés en ligne de commande et affiche leurs temps d'execution
    respectifs. \\

    Pour réaliser les différentes experiences, nous utiliserons un script nommé
    \verb+ experience.sh +

  \section{Le protocole}
    Les différentes experiences auront pour but d'étudier le nombre de défaut
    de page que cause un algorithme et le temps qu'il met à se terminer. \\

    Nous fixerons divers paramètres pour évaluer l'influence de chacun sur
    les deux variables que nous voulons étudier. \\

    Nous fixerons pour toutes les expériences le nombre d'accès car il n'est
    pas responsable de la variation de la proportion de defaut de page
    (Nous le fixerons à 100000).

    Nous allons commencer par fixer le nombre de case en RAM, puis nous ferons
    varier le nombre de pages.
    Il serait judicieux de procéder par puissance de 10.

  \section{Les résultats}
    RAM : 1 \\
    \begin{tabular}{|c|c|c|c|c|c|}
      \hline
        PAGES             & 10 & 100 & 1000 & 10000 & 100000 \\
      \hline
        Defauts FIFO    & 89960 & 98959 & 99896 &  99984 & 99999 \\
      \hline
        Defauts LRU     & 89960 & 98959 & 99896 & 99984 & 99999 \\
      \hline
        Defauts Horloge & 89960 & 98959 & 99896 & 99984 & 99999 \\
      \hline
        Defauts Optimal & 89960 & 98959 & 99896 & 99984 & 99999 \\
      \hline
    \end{tabular} \\

    RAM : 10 \\
    \begin{tabular}{|c|c|c|c|c|c|}
      \hline
        PAGES             & 10 & 100 & 1000 & 10000 & 100000 \\
      \hline
        Defauts FIFO    & $\times$ & 89987 & 98956 & 99892 & 99981 \\
      \hline
        Defauts LRU     & $\times$ & 90038 & 98988 & 99905 & 99983 \\
      \hline
        Defauts Horloge & $\times$ & 89960 & 98954 & 99892 & 99981\\
      \hline
        Defauts Optimal & $\times$ & 90006 & 98793 & 99175 & 99103 \\
      \hline
    \end{tabular} \\

    RAM : 100 \\
    \begin{tabular}{|c|c|c|c|c|c|}
      \hline
        PAGES             & 10 & 100 & 1000 & 10000 & 100000 \\
      \hline
        Defauts FIFO & $\times$ & $\times$ & 89964 & 98904 & 99802 \\
      \hline
        Defauts LRU & $\times$ & $\times$ & 89817 & 98895 & 99807 \\
      \hline
        Defauts Horloge & $\times$ & $\times$ & 89950 & 98901 & 99802 \\
      \hline
        Defauts Optimal & $\times$ & $\times$ & 89399 & 96507 & 96627 \\
      \hline
    \end{tabular} \\

    RAM : 1000 \\
    \begin{tabular}{|c|c|c|c|c|c|}
      \hline
        PAGES             & 10 & 100 & 1000 & 10000 & 100000 \\
      \hline
        Defauts FIFO & $\times$ & $\times$ & $\times$ & 89017 & 98041 \\
      \hline
        Defauts LRU & $\times$ & $\times$ & $\times$ & 89140 & 98009 \\
      \hline
        Defauts Horloge & $\times$ & $\times$ & $\times$ & 88997 & 98040 \\
      \hline
        Defauts Optimal & $\times$ & $\times$ & $\times$ & 83335 & 88549 \\
      \hline
    \end{tabular} \\

    \newpage
    RAM : 10000 \\
    \begin{tabular}{|c|c|c|c|c|c|}
      \hline
        PAGES             & 10 & 100 & 1000 & 10000 & 100000 \\
      \hline
        Defauts FIFO & $\times$ & $\times$ & $\times$ & $\times$ & 80622 \\
      \hline
        Defauts LRU & $\times$ & $\times$ & $\times$ & $\times$ & 80433 \\
      \hline
        Defauts Horloge & $\times$ & $\times$ & $\times$ & $\times$ & 80600 \\
      \hline
        Defauts Optimal & $\times$ & $\times$ & $\times$ & $\times$ & 60148 \\
      \hline
    \end{tabular}

    \subsection{Explication}
      Nous voyons très nettement se distinguer l'algorithme de bellady,
      surtout dans le dernier tableau où il diverge de près de 20000 defauts. \\

      Les autres algorithmes ne sembles pas se distinguer, ils semblent varier
      mais restent plus ou moins dans les mêmes ordres de grandeur.
\end{document}
